\paragraph*{Objective} \hfill \\
The objective of this experiment is to use use superposition to analyse the circuit. 

\paragraph*{Results} \hfill \\
Following values are given:
\begin{equation}
V=5V (With\,max\,current\,60mA) \,\,\,\,\,  I=60mA (With\,max\,voltage\,3V)  
\end{equation}
\begin{equation}
R_{1}=180\Omega \,\,\,\,\, R_{2}=18\Omega \,\,\,\,\, R_{3}=270\Omega \\
\end{equation}

The voltage that falls on $R_{3}$ is calculated using superposition.
Firs the voltage is set to 0; the circuit simplification process allows us to derive the following equations:

\begin{align*}
&R_{12}= \frac{R_{1}R_{2}}{R_{1}+R_{2}}=\frac{180\times 18}{180+18} = 16.36\Omega \\
&R_{5}= R_{12}+R_{2} = 16.36+18=34.36\Omega \\
&R_{P}= \frac{R_{5}R_{3}}{R_{5}+R_{3}}=\frac{34.36\times 270}{34.36+270}=30.48\Omega \\
\end{align*}
Next, from the equation: $ I=\frac{V_{a}}{V_{b}} $ \,\,\,\, the voltage $V_{a}$ can be calculated:\\
\begin{align*}
&V_{a}=I\times R_{P} = 0,06 \times 30.48 = 1.83 V 
\end{align*}



%add measured R3
\paragraph*{Conclusion} \hfill \\

