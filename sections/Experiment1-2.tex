\paragraph*{Objective} \hfill \\
The objective of this experiment is to learn how to build small circuits with resistors and to get familiar with analysing circuits using "device" and "connection" equations.

\paragraph*{Results} \hfill \\
Following resistors are given by: \hfill \\
\begin{align*}
& R_{1}=15k\Omega \,\,\,\,\,  R_{2}=22k\Omega \,\,\,\,\,  R_{3}=33k\Omega \,\,\,\,\, R_{4}=10k\Omega 
\end{align*}


\begin{description}
\item[a)] $ R=R_{1}+R_{2}+R_{3}= 15+22+33= 70k\Omega $ \hfill \\

\item[b)] $ R=R_{1} \parallel R_{2}+R_{3}= \frac{R_{1}R_{2}+R_{3}(R_{1}+R_{2})}{R_{1}+R_{2}}=\frac{330+33\times 37}{37}=\frac{330+1221}{37} \cong 41.92 k\Omega $ \hfill \\

\item[c)] $ R=(R_{1} \parallel R_{2})+R_{3}= \frac{R_{1}(R_{2}+R_{3})}{R_{1}+R_{2}+R_{3}}=\frac{15(22+33)}{15+22+33}=\frac{165}{14} \cong 11.8 k\Omega $ \hfill \\

\item[d)] $ R=R_{1}\parallel R_{2}+R_{3} \parallel R_{4} = \frac{R_{1}R_{2}}{R_{1}+R_{2}} +\frac{R_{3}R_{4}}{R_{3}+R_{4}} = \frac{330}{37}+\frac{330}{43} \cong 8.92+7.67 \cong 16.59 k\Omega $ \hfill \\

\item[e)] $ R=(R_{1}+R_{2}) \parallel (R_{3}+R_{4})= \frac{(R_{1}+R_{2})(R_{3}+R_{4})}{R_{1}+R_{2}+(R_{3}+R_{4}}= \frac{37\times 43}{80} \cong 19.89 k\Omega $ \hfill \\

\end{description}

\paragraph*{Conclusion} \hfill \\
