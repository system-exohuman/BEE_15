\paragraph*{Objective} \hfill \\
In this exercise we should learn how a resistor works and what its purpose is; how a resistor influences the voltage and the current, and the other way around.

\paragraph*{Results and procedure} \hfill\\
To find out what resistance a resistor has we looked at its colour coding, and calculated the resistance from there (brown, green, orange, gold $= 15k\Omega$ with 5\% tolerance) After identifying the resistor, it is smart to check if it actually works or not. This can be done by measuring the resistance with a multimeter.  
\begin{itemize}
\item Measurement on a $15k\Omega$ resistor shows: $14.98 k\Omega$
\end{itemize}
Conc. Looking at the colour coding we could see that the resistor has a tolerance of 5\%. Hence, the measured value is perfectly acceptable. In order to see how the resistor influences the current we set the voltage to 5V. To measure the voltage in the circuit, the multimeter was connected in parallel. To measure the current we put the multimeter in series. On the side, we made calculations to later be able to compare our measurement to the datasheet. The representation of the calculations and the measurements on voltage of 5V can be seen in (table 1).
\par "Step 2" was repeated, but this time for 11 different voltages ranging from -15 to 15V:
% add Table 1 [done] and 2 [to be]
% add calculations (V=iR)
% add plot [done]

\paragraph*{Conclusion} \hfill \\