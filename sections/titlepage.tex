\pdfbookmark[0]{English title page}{label:titlepage_en}
\aautitlepage{%
  \englishprojectinfo{
    Multi-level liquid monitoring in storage tanks
  }{%
    Scientific Theme %theme
  }{%
    Spring Semester 2015 %project period
  }{%
    H107 % project group
  }{%
    %list of group members
	Andris Lipenitis\\
	Benjamin Lentz Olesen\\
	Edgar Lipenitis\\
	Kacper Kisiel\\
	Lyuboslav Veselinov\\
	Michal Damian Budzinski
  }{%
    %list of supervisors
	Akbar Hussain\\
	Torben Rosenørn
  }{%
    3 % number of printed copies
  }{%
    \today % date of completion
  }%
}{%department and address
  \textbf{School of Information and Communication Technology}\\
  Niels Bohrs Vej 8\\
  DK-6700 Esbjerg\\
  \href{http://sict.aau.dk}{http://sict.aau.dk}
}{% the abstract
	Larger fuel storage tanks often contain a layer of water at the bottom of the tank. Companies want to extract the water because it can mix with the fuel once it is pumped out, thus affecting the quality of it and influencing the lifespan of all kinds of fuel engines. Analyses of these storage tanks have suggested the presence of different water level monitoring systems, but none of them are automated. To improve their safety, durability and reliability, we combined the ultrasonic sensor already in use and a capacitive sensor. This paper presents a simple, cheap, reliable and automated sensor system to keep track of the water level in the tank in real-time.

In conclusion, our prototype can be used for gathering data, planning maintenance and can also be used for help when extracting water from the tank. In the future, the range of implementations will need to be considered, since the system is not limited to only operate with one specific set of fluids (In this case: water and fuel). It will be able to measure and monitor the levels of different mediums as long as they have distinct densities. 
}