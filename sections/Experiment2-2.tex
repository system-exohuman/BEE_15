\paragraph*{Objective} \hfill \\
The objective of this experiment is to use Thevenin and Norton Theories and find their equivelents, by experimend and also by calculating.
\paragraph*{Results and procedure} \hfill\\
Following values are given by : \\
\begin{align*}
&V=20V	&	R_{1}&= 1k\Omega	&	R_{2}&= 100k\Omega	&	R_{3}&= 4.7k\Omega	&	R_{4}&= 1.8k\Omega
\end{align*}

\begin{flushleft}
The Thevenin Equivalent is measured to: $ R_{th}= 3k\Omega $. Resistance ($ R_{th} $) is calculated by:
\end{flushleft}

\begin{align*}
R_{th}&= (R_{1} \parallel R_{3})+(R_{2} \parallel R_{4}) = \frac{R_{1}R_{3}}{R_{1}+R_{3}}+ \frac{R_{2}R_{4}}{R_{2}+R_{4}} = \frac{1 \times 4.7}{1+4.7} + \frac{100 \times 1.8}{100+ 1.8} \\
&=2.59 \cong 2.6k\Omega
\end{align*}

\begin{flushleft}
Voltage ($ V_{th} $) is calculated by:
\end{flushleft}
\begin{align*}
V_{th}&=V(\frac{R_{3}}{R_{1}+R_{3}}-\frac{R_{4}}{R_{2}-R_{4}})
&=20(\frac{4.7}{1+4.7}-\frac{1.8}{100+1.8})	
&=20 \times 0.8 =16.1V
\end{align*}
%draw circuit TH

\begin{flushleft}
Norton equivalent current($ I_{n} $) is measured to: $ I_{n}= 5.98mA $. Resistance ($ R_{th} $) is calculated by:
\end{flushleft}

\begin{align*}
&I_{n}= \frac{V_{th}}{R_{th}}= \frac{16.1}{2.6}= 6.2mA
\end{align*}
%Draw norton NC
\paragraph*{Conclusion} \hfill \\