In this exercise we should learn how a resistor works and what is its purpose. We will see how a resistor influences the voltage and the current, and the other way around.

\subsection{Experiment results and procedure}
\begin{enumerate}
\item There are a few ways to find out what resistance a resistor has. One can take a resistor from a accordingly named box, and make experiments. This not the way to be sure though. A better option is looking at its colour coding, and calculate the resistance from there. After identifying the resistor, it is smart to check if it is working or not. This can be done by measuring the resistance on a mult  imeter.  The multimeter is connected to the ground and the voltage input. The function is set to measure resistance (Marked by a symbol $ \Omega $).
\begin{itemize}
\item Measurement on a $15k\Omega$ resistor shows: $14.98 k\Omega$
\end{itemize}
Conc. Looking at the colour coding we could see that the resistor has a tolerance of 1\%. Hence the measured value is perfectly acceptable. 
 
\item 2.	The circuit was built using a power supply, the resistor, multimeter, few cables and wires. In order to see how the resistor influences the current we in the first place set the voltage to 5V. To measure the voltage in the circuit, the multimeter is connected in parallel. To measure the current we put the multimeter in series. On the side, we make calculations to later be able to compare our measurement to the “real or accepted values”. The representation of the calculations and the measurements on voltage of 5V can be seen in the section below.

\item we repeat step 2, but now for 11 different voltages ranging from -15 to 15V. Table(1) presents the “real or accepted values”  that were calculated and in the table(2) our findings are presented.
% add Table 1 and 2
% add calculations (V=iR)

\item We draw two plots that represent I-V characteristics of this resistor.
% add plot
\end{enumerate}

\subsection{Conclusion}